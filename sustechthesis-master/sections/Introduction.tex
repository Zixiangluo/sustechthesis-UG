% !Mode:: "TeX:UTF-8"
% !TEX program  = xelatex
\section{Introduction}

\subsection{single cell RNA sequencing}

In the past decades, bulk cell RNA sequencing were extensively used to study the gene expression features at the tissue level. However, it only reflects the averaged expression level of cells in a tissue. As the rapid growth of single cell RNA sequencing (scRNA-seq) technologies, it enables researchers to study the gene expression patterns at single cell level. It greatly aids researchers to identify molecular heterogeneity and better understand biological mechanisms \cite{shapiro2013single}. For example, the single-cell sequencing of cancer cells reveals the mutations on each sub-population of cells and thus helps in constructing the cancer development trajectory. 

\subsection{Dimensionality reduction of scRNA-seq data}

The single cell sequencing data usually have more than two thousand dimensions, which makes it difficult to be analyzed directly. Thus, finding a meaningful low dimensional representation to represent those cell states is a necessary step in scRNA-seq data analyzing. In fact, those genes are highly regulated and coexpressed, so that there are only limited number of cell states in a group of cells. It is assumed that those data form a low dimensional manifold in a high dimensional space. The process of dimensionality reduction can extracts the crucial information in the original high dimensional data and represent them as a low dimensional structure.  It plays an important role in data visualization and data preprocessing which allows efficient downstream analysis such as trajectory inference \cite{Saelens2019}, cell clustering \cite{weber2016comparison} and cell sub-population identification \cite{Hwang2018}. 

For example, Seurat uses PCA \cite{Abdi2010} to reduce dimensionality before clustering \cite{Satija2015}. Destiny \cite{angerer2016destiny} uses diffusion map before clustering. Monocle \cite{Qiu2017} performs uniform manifold approximation and projection (UMAP) \cite{McInnes2018} or independent component analysis (ICA) \cite{hyvarinen2000independent} before inferencing cell trajectories. 

\subsection{Dropout events}

There are many traditional dimensionality reduction methods that work well for bulk cell sequencing data. Some of them seek to preserve global structure such as PCA \cite{Abdi2010} and MDS \cite{Kruskal1964}, while others prefer to keep local structures such as LLE \cite{roweis2000nonlinear} and tSNE \cite{Kobak2019}.

However, the single cell sequencing data have more noise than bulk cell data because of the low capture efficiency in single cell sequencing \cite{Peng2019}. Thus, single cell data have excessive dropout events which means many data points that should have bigger value are presented as zero or near zero. The traditional methods are inefficient because of those noises.  

To address this challenge, many methods that are specific designed to single cell expression data are developed. There are three main categories. The first strategy is to probabilistically model the conditional dropout events such as ZIFA \cite{Pierson2015}. It uses a double exponential function to mimic the dropout probability and uses EM algorithm \cite{mclachlan2007algorithm} to identify the model parameters.

The second strategy is to impute those dropouts by borrowing values of the same gene in other cells. Those cells are chose based on the genes that are unlikely to be affected by dropout events. For example, RESCUE \cite{Tracy2019} and scImpute \cite{Li2018}. 

Moreover, one kind of approaches make use of deep learning technologies. Most of them build the model based on autoencoder or variational autoencoder and use the bottleneck layer as the dimensionality reduction results. For example, VASC \cite{Wang2018} adds the dropout model in ZIFA to a VAE model to enable it to model the dropout events. DCA \cite{Eraslan2019a} uses autoencoder to impute the missing values by assuming the zero-inflated negative binomial model \cite{Hafemeister2019}. scScope \cite{Deng2019} uses a recurrent neuron network to impute and reduce the dimensionality. 

In this study, I compared the performance on visualization and clustering of several popular dimensionality reduction methods on scRNA-seq data. Also, based on the structure of VAE \cite{Kingma2014} and VASC \cite{Wang2018}, I improved the structure of VASC and proposed a new model, zero-inflated variational autoencoder (ZIVA) to analyze and visualize the single cell RNA sequencing data. It uses the gumbel softmax and double exponential function to model the dropout events. I compared the performance of ZIVA and some other popular methods and showed that ZIVA has good performance on cell clustering and cell trajectory inference.

