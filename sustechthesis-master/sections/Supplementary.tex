% !Mode:: "TeX:UTF-8"
% !TEX program  = xelatex
\section{Supplementary materials}
\noindent\emph{Datasets} \\
To test the performance of the ZIVA method, seven datasets were used in this study. The first six datasets were downloaded from the Hemberg group (https://hemberg-lab. github.io/scRNA.seq.datasets). The seventh dataset was obtained from GEO database with accession number GSE87375.

\vspace{0.5cm}
\noindent\emph{Software and packages} \\
The PCA, tSNE and UMAP was performed using the Seurat R package from \url{https://satijalab.org/seurat/}, following the tutorial from \url{https://satijalab.org/seurat/v3.1/pbmc3k_tutorial.html}. It first performs normalization and scaling to the data. Then, run the three dimensionality reduction methods.\\
The ZIFA, VASC, Ivis codes was download and ran after normalization and scaling of the count data.\\
The ZIFA was performed using the package from \url{https://github.com/epierson9/ZIFA}, the block algorithm was used.\\
The codes of VASC was downloaded from \url{https://github.com/wang-research/VASC}.\\
The Ivis package was downloaded following the \url{https://github.com/beringresearch/ivis}.\\

\vspace{0.5cm}
\noindent\emph{Supplementary Tables} \\
\begin{table}[htb!]
\centering
\caption{scRNA-seq datasets used in this study}
\label{datasets}
\resizebox{13cm}{!}{
\begin{tabular}{llllllll}
\hline
No. & Dataset       & No. of cells & No. of genes & No. of types & Protocol   & Tissue       & ref \\ \hline
1   & Pollen        & 301          & 23730        & 11           & SMARTer    & Multiple     & \cite{pollen2014low}   \\
2   & Jin           & 2610         & 2274         & 19           & inDrop     & HSPC         & Jin \\
3   & Yan           & 90           & 20214        & 6            & Tang       & Embryo Devel &  \cite{yan2013single}  \\
4   & Camp          & 777          & 19020        & 7            & SMARTer    & Liver        & \cite{camp2017multilineage} \\
5   & Goolam        & 124          & 41480        & 5            & Smart-seq2 & Embryo devel &  \cite{goolam2016heterogeneity}\\
6   & Kolodziejczyk & 704          & 38653        & 9            & SMARTer    & Stem Cells   &    \cite{kolodziejczyk2015single} \\ \hline
\end{tabular}}
\end{table}


\begin{table}[htb!]
\centering
\caption{NMI of clustering performance of seven methods on six datasets}
\label{nmiall}
\resizebox{10cm}{!}{
\begin{tabular}{llllllll}
\hline
  & PCA  & tSNE & UMAP & ZIFA & VASC & Ivis & ZIVA \\ \hline
1 & 0.88 & 0.92 & 0.92 & 0.60 & 0.79 & 0.86 & 0.84 \\
2 & 0.57 & 0.67 & 0.71 & 0.59 & 0.65 & 0.65 & 0.67 \\
3 & 0.79 & 0.87 & 0.91 & 0.79 & 0.75 & 0.81 & 0.79 \\
4 & 0.70 & 0.78 & 0.81 & 0.60 & 0.72 & 0.81 & 0.66 \\
5 & 0.86 & 0.73 & 0.73 & 0.76 & 0.48 & 0.94 & 0.94 \\
6 & 0.70 & 0.84 & 0.87 & 0.66 & 0.25 & 0.65 & 0.60 \\ \hline
\end{tabular}}
\end{table}

\begin{table}[htb!]
\centering
\caption{NMI of clustering performance of seven methods on six datasets}
\label{ariall}
\resizebox{10cm}{!}{
\begin{tabular}{llllllll}
\hline
  & PCA  & tSNE & UMAP & ZIFA & VASC & Ivis & ZIVA \\ \hline
1 & 0.79 & 0.84 & 0.84 & 0.44 & 0.65 & 0.80 & 0.73 \\
2 & 0.34 & 0.41 & 0.48 & 0.36 & 0.39 & 0.39 & 0.43 \\
3 & 0.63 & 0.78 & 0.90 & 0.63 & 0.60 & 0.74 & 0.64 \\
4 & 0.56 & 0.61 & 0.64 & 0.42 & 0.54 & 0.70 & 0.47 \\
5 & 0.71 & 0.54 & 0.54 & 0.75 & 0.45 & 0.97 & 0.98 \\
6 & 0.48 & 0.74 & 0.77 & 0.48 & 0.11 & 0.48 & 0.43 \\ \hline
\end{tabular}}
\end{table}

\begin{table}[htb!]
\centering
\caption{NMI and ARI of clustering performance of Autoencoder, VAE and ZIVA.}
\label{t3ae}
\begin{tabular}{lllllll}
\hline
  & \multicolumn{3}{l}{NMI} & \multicolumn{3}{l}{ARI} \\ \hline
  & AE     & VAE    & ZIVA  & AE     & VAE    & ZIVA  \\ \hline
1 & 0.82   & 0.89   & 0.84  & 0.73   & 0.87   & 0.75  \\
2 & 0.64   & 0.67   & 0.67  & 0.37   & 0.41   & 0.43  \\
3 & 0.76   & 0.77   & 0.79  & 0.63   & 0.74   & 0.64  \\
4 & 0.67   & 0.69   & 0.69  & 0.48   & 0.52   & 0.51  \\
5 & 0.67   & 0.72   & 0.94  & 0.72   & 0.54   & 0.98  \\
6 & 0.58   & 0.51   & 0.60  & 0.40   & 0.33   & 0.43  \\ \hline
\end{tabular}
\end{table}

\begin{table}[htb!]
\centering
\caption{NMI and ARI of clustering performance of ZIVA under different dropout model.}
\label{tnbmm}
\begin{tabular}{lllll}
\hline
  & \multicolumn{2}{l}{NMI} & \multicolumn{2}{l}{ARI} \\ \hline
  & NB         & MM         & NB         & MM         \\ \hline
1 & 0.84       & 0.84       & 0.75       & 0.78       \\
2 & 0.67       & 0.63       & 0.41       & 0.34       \\
3 & 0.79       & 0.87       & 0.64       & 0.78       \\
4 & 0.69       & 0.68       & 0.52       & 0.51       \\
5 & 0.94       & 0.83       & 0.98       & 0.67       \\
6 & 0.61       & 0.64       & 0.44       & 0.47       \\ \hline
\end{tabular}
\end{table}


\begin{table}[htb!]
\centering
\caption{NMI and ARI of clustering performance of ZIVA under different latent dimensions.}
\label{dimt}
\begin{tabular}{lllllllll}
\hline
    & 2    & 4    & 6    & 8    & 10   & 15   & 20   & All  \\ \hline
NMI & 0.72 & 0.69 & 0.70 & 0.70 & 0.69 & 0.69 & 0.69 & 0.73 \\
ARI & 0.51 & 0.43 & 0.46 & 0.46 & 0.46 & 0.45 & 0.44 & 0.52 \\ \hline
\end{tabular}
\end{table}