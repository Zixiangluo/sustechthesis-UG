% !Mode:: "TeX:UTF-8"
% !TEX program  = xelatex
\begin{中文摘要}{单细胞测序,降维,变分自编码器}
  单细胞测序技术使我们可以有效的发现组织中细胞间的异质性。降维对聚类和发育轨迹推断等后续分析有着关键的作用。在本研究中,我们提出了一种新的对单细胞测序数据降维的方法:针对零膨胀数据的变分自编码器(ZIVA)。它可以模拟单细胞测序数据中的数据丢失现象从而可以准确的可视化数据以及判断细胞类型。我们还系统的比较了我们的模型和其他六个常用的方法在六个数据集上的表现。ZIVA在多数情况下都有良好的可视化效果。
\end{中文摘要}

\begin{英文摘要}{scRNA-seq, Dimensionality reduction, Variational autoencoder}
  Single cell sequencing technology enable us to effectively identify cellular heterogeneity in healthy and disease tissues. Dimensionality reduction of high dimensional expression data is a critical step for the downstream analysis such as cell clustering and lineage inference. In this study, we proposed a new method for dimensionality reduction of scRNA-seq data: zero-inflated variational autoencoder (ZIVA). It can model the dropout events in scRNA-seq data and accurately identify cell sub-populations. We systematically compared our model with six other popular methods on six datasets. It shows that ZIVA perform well in most cases.
\end{英文摘要}
